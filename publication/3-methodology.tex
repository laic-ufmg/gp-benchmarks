\section{Methodology}

As previously explained, the main objective of this paper is to characterize a set of GP benchmarks in symbolic regression according to the relationship of their attributes and the error generated by the GP.

\begin{enumerate}
\item We looked at the datasets that have been used to evaluate GP methods at GECCO papers in the last five years.
\item We extracted from these datasets a set of meta-features in order to characterize them. It is important to mention that the are of meta-learning has extensively study attributes to characterize that, and this is still an open problem \cite{}. We have chosen 5 meta-attributes and generated them for both the training and test sets.
\item We run a canonical GP method in the selected datasets, and recorded the root mean squared error (RMSE) for each dataset.
\item We created a meta data-set, where each attribute represents a meta-feature and is associated with the RMSE generated for that respective dataset.
\item We used this dataset to determine which attributes were more correlated to the RMSE obtained.
\end{enumerate}


\subsection{Datasets}

The XX papers working with symbolic regression problems published at GECCO in the past five years used a total of XX different datasets.
We could not find the description two synthetic datasets---
Sext \cite{krawiec2014behavioral} and Nguyen-12 \cite{krawiec2014behavioral,liskowski2017discovery}---and we could not find XXXX real-world datasets available on line---Dow chemical \cite{nicolau2016managing}, Plasma protein binding level (PPB) \cite{castelli2015geometric,oliveira2016dispersion,goncalves2017unsure}, Tower Data \cite{lacava2015genetic,lacava2016epsilon,oliveira2016dispersion}, NOX  \cite{arnaldo2014multiple,arnaldo2015building}, Wind (WND) \cite{lacava2016epsilon}, Median lethal dose, toxicity (LD50) \cite{castelli2015geometric,goncalves2017unsure} and Human oral bioavailability  \cite{castelli2015geometric,dick2015reexamination,oliveira2016dispersion,goncalves2017unsure}.
We ended up with XX datasets, listed in Table~\ref{tab:datasets}

\input{3a-datasets}

\subsection{Meta-attributes}

We selected a set of five meta-attributes to work with:
\begin{enumerate}
\item Number of attributes;
\item Number of instances; 
\item Target attribute skewness;
\item Standard deviation of the target attribute.
\end{enumerate}

The final meta dataset had XX attributes and XX examples.


\subsection{Data Analysis}

The meta dataset was given to a Random Forest regressor \cite{}, which generated a ranking of the most relevant meta-attributes to predict the RMSE of the problem.
We also run a dimensionality reduction method, namely PCA, and analysed the correlation of the dataset representation to its RMSE.