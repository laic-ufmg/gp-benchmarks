\documentclass[journal]{IEEEtran}
\usepackage[T1]{fontenc}
\usepackage[utf8]{inputenc}

\usepackage{cite}
\usepackage{amsmath}
\usepackage{url}

\usepackage[linesnumbered,ruled,vlined]{algorithm2e}
\usepackage{amsfonts}
\usepackage{caption}
\usepackage{color}
%\usepackage{showframe}
\usepackage{soul} % underlining
\usepackage[dvipsnames]{xcolor}
\usepackage{subcaption}
\usepackage[numbers]{natbib}

\usepackage{multirow}
\usepackage{booktabs}

% correct bad hyphenation here
\hyphenation{op-tical net-works semi-conduc-tor}

\colorlet{dark-green}{green!70!black}

\newcommand{\lo}[1]{\textcolor{blue}{[#1]}} 
\newcommand{\lf}[1]{\textcolor{dark-green}{[#1]}}
\newcommand{\jf}[1]{\textcolor{orange}{[#1]}}
\newcommand{\gi}[1]{\textcolor{red}{[#1]}}


\begin{document}

\setul{0.5ex}{0.3ex}
\definecolor{Green}{rgb}{0,1,0}
\setulcolor{dark-green}

\title{Genetic Programming Benchmarks: A Data Science Approach}
%
%
% author names and IEEE memberships
% note positions of commas and nonbreaking spaces ( ~ ) LaTeX will not break
% a structure at a ~ so this keeps an author's name from being broken across
% two lines.
% use \thanks{} to gain access to the first footnote area
% a separate \thanks must be used for each paragraph as LaTeX2e's \thanks
% was not built to handle multiple paragraphs
%

\author{Luis~F.~Miranda,
        Luiz~Otavio~V.~B.~Oliveira,
        Joao~Francisco~B.~S.~Martins,
        and~Gisele~L.~Pappa,~\IEEEmembership{Member,~IEEE,}% <-this % stops a space
\thanks{Luis F. Miranda, Luiz Otavio V. B. Oliveira, Joao Francisco B. S. Martins and Gisele L. Pappa are with the Department of Computer Science, Universidade Federal de Minas Gerais, Belo Horizonte, Brazil (e-mail: \{luisfmiranda, luizvbo, joaofbsm, glpappa\}@dcc.ufmg.br)}% <-this % stops a space
\thanks{Manuscript received July 20, 2017; revised October 17, 2017.}}

% note the % following the last \IEEEmembership and also \thanks - 
% these prevent an unwanted space from occurring between the last author name
% and the end of the author line. i.e., if you had this:
% 
% \author{....lastname \thanks{...} \thanks{...} }
%                     ^------------^------------^----Do not want these spaces!
%
% a space would be appended to the last name and could cause every name on that
% line to be shifted left slightly. This is one of those "LaTeX things". For
% instance, "\textbf{A} \textbf{B}" will typeset as "A B" not "AB". To get
% "AB" then you have to do: "\textbf{A}\textbf{B}"
% \thanks is no different in this regard, so shield the last } of each \thanks
% that ends a line with a % and do not let a space in before the next \thanks.
% Spaces after \IEEEmembership other than the last one are OK (and needed) as
% you are supposed to have spaces between the names. For what it is worth,
% this is a minor point as most people would not even notice if the said evil
% space somehow managed to creep in.



% The paper headers
\markboth{Journal of \LaTeX\ Class Files,~Vol.~13, No.~9, September~2014}%
{Shell \MakeLowercase{\textit{et al.}}: Bare Demo of IEEEtran.cls for Journals}
% The only time the second header will appear is for the odd numbered pages
% after the title page when using the twoside option.
% 
% *** Note that you probably will NOT want to include the author's ***
% *** name in the headers of peer review papers.                   ***
% You can use \ifCLASSOPTIONpeerreview for conditional compilation here if
% you desire.




% If you want to put a publisher's ID mark on the page you can do it like
% this:
%\IEEEpubid{0000--0000/00\$00.00~\copyright~2014 IEEE}
% Remember, if you use this you must call \IEEEpubidadjcol in the second
% column for its text to clear the IEEEpubid mark.



% use for special paper notices
%\IEEEspecialpapernotice{(Invited Paper)}



% make the title area
\maketitle

% As a general rule, do not put math, special symbols or citations
% in the abstract or keywords.
\begin{abstract}
The abstract goes here.
\end{abstract}

% Note that keywords are not normally used for peerreview papers.
\begin{IEEEkeywords}
IEEEtran, journal, \LaTeX, paper, template.
\end{IEEEkeywords}

\IEEEpeerreviewmaketitle

\section{Introduction}

The quest for better Genetic Programming (GP) based algorithms---by overcoming known drawbacks, proposing new paradigms or hybridizing existing ones---is as important as the search for better ways of evaluating these methods and comparing them under different aspects. It is necessary to consistently and efficiently identify the scenarios where the new algorithm excels and how it compares to its predecessors.

\bibliographystyle{IEEEtranN}
\bibliography{references}


\end{document}