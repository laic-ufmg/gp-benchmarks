\section{Methodology}

As previously explained, the main objective of this paper is to characterize a set of GP benchmarks in symbolic regression according to the relationship of their attributes and the error generated by the GP.

\begin{enumerate}
\item We looked at the datasets that have been used to evaluate GP methods at GECCO papers in the last five years.
\item We extracted from these datasets a set of meta-features in order to characterize them. It is important to mention that the are of meta-learning has extensively study attributes to characterize that, and this is still an open problem \cite{}. We have chosen 5 meta-attributes and generated them for both the training and test sets.
\item We run a canonical GP method in the selected datasets, and recorded the root mean squared error (RMSE) for each dataset.
\item We created a meta data-set, where each attribute represents a meta-feature and is associated with the RMSE generated for that respective dataset.
\item We used this dataset to determine which attributes were more correlated to the RMSE obtained.
\end{enumerate}


\subsection{Datasets}

The XX papers working with symbolic regression problems published at GECCO in the past five years used a total of XX different datasets.
We could not find the description two synthetic datasets---
Sext \cite{krawiec2014behavioral} and Nguyen-12 \cite{krawiec2014behavioral,liskowski2017discovery}---and we could not find XXXX real-world datasets available on line---Dow chemical \cite{nicolau2016managing}, Plasma protein binding level (PPB) \cite{castelli2015geometric,oliveira2016dispersion,goncalves2017unsure}, Tower Data \cite{lacava2015genetic,lacava2016epsilon,oliveira2016dispersion}, NOX  \cite{arnaldo2014multiple,arnaldo2015building}, Wind (WND) \cite{lacava2016epsilon}, Median lethal dose, toxicity (LD50) \cite{castelli2015geometric,goncalves2017unsure} and Human oral bioavailability  \cite{castelli2015geometric,dick2015reexamination,oliveira2016dispersion,goncalves2017unsure}.
We ended up with XX datasets, listed in Table~\ref{tab:datasets}

\begin{table*}[htbp]
\scriptsize
\caption{Datasets from the last five GECCO}
\begin{center}
\begin{tabular}{llll}

Meier-3  & \cite{meier2013accelerating} & $(x_1^2*x_2^2)/(x_1+x_2)$ & Training: $U[-1,1,50]$ \\ 
Meier-4  & \cite{meier2013accelerating} & $x_1^5/x_2^3$ &  Test: Undefined \\ 
Nonic & \cite{krawiec2013approximating,szubert2016reducing} & $\sum\limits_{i=1}^9 x_1^i$ & Train.: $E[-1,1,20]$ / Test: $U[-1,1,20]$ \\ 
Sine & \cite{mcphee2015impact} & $sin(x_1)+sin(x_1+x_1^2)$ & Train.: $E[0,6.2,0.1]$ / Test: Undef. \\ 
Burks & \cite{burks2015efficient,szubert2016reducing} & $4*x_1^4+3*x_1^3+2*x_1^2+x_1$ & Train.: $U[-1,1,20]$ / Test: Undef. \\ 
R1 & \cite{krawiec2013approximating,szubert2016reducing,liskowski2017discovery} & $(x_1+1)^3/(x_1^2-x_1+1)$ & Training: $E[-1,1,20]$ / Test: $U[-1,1,20]$ \\ 
R2 & \cite{krawiec2013approximating,szubert2016reducing,liskowski2017discovery} & $(x_1^5-3*x_1^3+1)/(x_1^2+1)$ & Train.: $E[-1,1,20]$ / $Test: U[-1,1,20]$ \\ 
R3 & \cite{liskowski2017discovery} & $(x_1^6+x_1^5)/(x_1^4+x_1^3+x_1^2+x_1+1)$ & Training: $E[-1,1,20]$ / Test: $U[-1,1,20]$  \\ 

Poly-10 & \cite{medernach2016new} & $x_1*x_2+x_3*x_4+x_5*x_6+x_1*x_7*x_9+x_3*x_6*x_10$ & Train.: $U[0,1,330]^{9}$ / Test: $U[0,1,170]^{9}$ \\ 
Koza-2 & \cite{meier2013accelerating} & $x_1^5-2*x_1^3+x_1$ & $U(-1, 1, 20)$ \\ 
Koza-3 & \cite{meier2013accelerating, harada2014asynchronously} & $x_1^6-2*x_1^4+x_1^2$ & $U(-1, 1, 20)$ \\ 
Korns-1 & \cite{worm2013prioritized} & $1.57+24.3*x_4$ &  \\  
Korns-4 & \cite{worm2013prioritized} & $-2.3+0.13*sin(x_3)$ &  \\ 
Korns-7 & \cite{worm2013prioritized} & $213.80940889*(1-e^{-0.54723748542*x_1})$ &  \\ 
Korns-11 & \cite{worm2013prioritized} & $6.87+11*cos(7.23*x_1^3)$ &  \\ 
Korns-12 & \cite{worm2013prioritized} & $2-2.1*cos(9.8*x_1)*sin(1.3*x_5)$ &  \\ 

% Add explanation (undefined instaces) regarding the following Korns

Korns-2 & \cite{worm2013prioritized} & $0.23+14.2*(x_4+x_2)/(3*x_5)$ &  \\ 
Korns-3 & \cite{worm2013prioritized,sotto2017probabilistic} & $-5.41+4.9*(x_4-x_1+x_2/x_5)/(3*x_5)$ &  \\
Korns-5 & \cite{worm2013prioritized,sotto2017probabilistic} & $3+2.13*ln(x_5)$ &  \\ 
Korns-6 & \cite{worm2013prioritized} & $1.3+0.13*\sqrt(x_1)$ &  \\ 
Korns-8 & \cite{worm2013prioritized} & $6.87+11*\sqrt{7.23*x_1*x_4*x_5}$ &  \\ 
Korns-9 & \cite{worm2013prioritized} & $(\sqrt{x_1}/ln(x_2))*(e^{x_3}/x_4^2)$ &  \\ 
Korns-10 & \cite{worm2013prioritized} & $0.81+24.3*(2*x_2+3*x_3^2)/(4*x_4^3+5*x_5^4)$ &  \\ 

Vladislavleva-1 (Vlad1) & \cite{oliveira2016dispersion,miranda2017how,liskowski2017discovery,thuong2017combining} & $e^{-(x_1-1)^2}/(1.2+(x_2-2.5)^2)$ &  \\ 
Vladislavleva-2 (Vlad2) & \cite{miranda2017how} & $e^{-x_1}*x_1^3*(cos x_1*sin x_1)*(cos x_1*sin^2 x_1 – 1)$ &  \\ 
Vladislavleva-3 (Vlad3) & \cite{miranda2017how} & $e^{-x_1}*x_1^3*(cos x_1*sin x_1)*(cos x_1*sin^2 x_1 – 1)*(x_2-5)$ &  \\ 
Vladislavleva-4 (Uball5D) & \cite{lacava2015genetic,medernach2016new,nicolau2016managing,lacava2016epsilon,oliveira2016dispersion} & $10/(5+\sum\limits_{i=1}^5 (x_i-3)^2)$ &  \\ 
Vladislavleva-5 (Vlad5) & \cite{chen2016improving,thuong2017combining} & $30*(x_1-1)*(x_3-1)/((x_1-10)*x_2^2)$ &  \\ 
Vladislavleva-6 (Vlad6) & \cite{chen2016improving,thuong2017combining} & $6*sin(x_1)*cos(x_2)$ &  \\ 
Vladislavleva-7 (Vlad7) & \cite{miranda2017how} & $(x_1-3)*(x_2-3)+2*sin((x_1-4)*(x_2-4))$ &  \\ 
Vladislavleva-8 (Vlad8) & \cite{chen2016improving,thuong2017combining} & $((x_1-3)^4+(x_2-3)^3-(x_2-3))/((x_2-2)^4+10)$ &  \\ 
Pagie-1 & \cite{mcphee2015impact,lacava2015genetic,liskowski2017discovery} & $1/(1+x_1^{-4})+1/(1+x_2^{-4})$ &  \\ 
Keijzer-1 & \cite{krawiec2013approximating,krawiec2014behavioral,demelo2014kaizen,miranda2017how,liskowski2017discovery} & $0.3*x^1*sin(2*\pi*x_1)$ &  \\ 
Keijzer-2 & \cite{demelo2014kaizen,miranda2017how} & $0.3*x^1*sin(2*\pi*x_1)$ &  \\ 
Keijzer-3 & \cite{demelo2014kaizen,miranda2017how} & $0.3*x^1*sin(2*\pi*x_1)$ &  \\ 
Keijzer-4 & \cite{krawiec2013approximating,krawiec2014behavioral,demelo2014kaizen,szubert2016reducing,chen2016improving,sotto2017probabilistic,miranda2017how,liskowski2017discovery} & $x_1^3*e^{-x_1}*cos(x_1)*sin(x_1)*(sin^2(x_1)*cos(x_1)-1)$ &  \\ 
Keijzer-5 & \cite{krawiec2014behavioral,demelo2014kaizen,sotto2017probabilistic,oliveira2016dispersion} & $30*x_1*x_3/((x_1-10)*x_2^2)$ &  \\ 
Keijzer-6 & \cite{demelo2014kaizen,lacava2015genetic,nicolau2016managing,oliveira2016dispersion,miranda2017how,medvet2017evolvability} & $\sum\limits_{i=1}^{x_1}i$ &  \\ 
Keijzer-7 & \cite{demelo2014kaizen,oliveira2016dispersion,miranda2017how} & $ln x_1$ &  \\ 
Keijzer-8 & \cite{demelo2014kaizen,miranda2017how,liskowski2017discovery} & $\sqrt(x_1)$ &  \\ 
Keijzer-9 & \cite{demelo2014kaizen,miranda2017how} & $arcsinh(x_1) = ln(x_1+\sqrt{x_1^2+1})$ &  \\ 
Keijzer-10 & \cite{wieloch2013running,demelo2014kaizen,thuong2017combining} & $x_1^{x_2}$ &  \\ 
Keijzer-11 & \cite{krawiec2014behavioral,demelo2014kaizen,chen2016improving,thuong2017combining} & $x_1*x_2+sin((x_1-1)*(x_2-1))$ &  \\ 
Keijzer-12 & \cite{wieloch2013running,krawiec2014behavioral,demelo2014kaizen,chen2016improving,thuong2017combining} & $x_1^4-x_1^3+(x_2^2/2)-x_2$ &  \\ 
Keijzer-13 & \cite{krawiec2014behavioral,demelo2014kaizen,thuong2017combining} & $6*sin(x_1)*cos(x_2)$ &  \\ 
Keijzer-14 & \cite{krawiec2014behavioral,demelo2014kaizen,chen2016improving,liskowski2017discovery,thuong2017combining} & $8/(2+x_1^2+x_2^2)$ &  \\ 
Keijzer-15 & \cite{krawiec2014behavioral,demelo2014kaizen,chen2016improving,liskowski2017discovery,thuong2017combining} & $(x_1^3/5)+(x_2^3/2)-x_2-x_1$ &  \\ 
Nguyen-1 & \cite{worm2013prioritized,demelo2014kaizen,sotto2017probabilistic} & $x_1^3+x_1^2+x_1$ &  \\ 
Nguyen-2 & \cite{worm2013prioritized,lopes2013gearnet,harada2014asynchronously,demelo2014kaizen,whigham2015examining,sotto2017probabilistic,medvet2017evolvability} & $x_1^4+x_1^3+x_1^2+x_1$ &  \\ 
Nguyen-3 & \cite{worm2013prioritized,wieloch2013running,krawiec2014behavioral,demelo2014kaizen,sotto2017probabilistic,liskowski2017discovery} & $x_1^5+x_1^4+x_1^3+x_1^2+x_1$ &  \\ 
Nguyen-4 & \cite{worm2013prioritized,wieloch2013running,krawiec2014behavioral,demelo2014kaizen,sotto2017probabilistic,liskowski2017discovery} & $x_1^6+x_1^5+x_1^4+x_1^3+x_1^2+x_1$ &  \\ 
Nguyen-5 & \cite{worm2013prioritized,wieloch2013running,harada2014asynchronously,krawiec2014behavioral,demelo2014kaizen,liskowski2017discovery} & $sin(x_1^2)*cos(x_1)-1$ &  \\ 
Nguyen-6 & \cite{worm2013prioritized,wieloch2013running,krawiec2014behavioral,demelo2014kaizen,sotto2017probabilistic,liskowski2017discovery} & $sin(x_1)+sin(x_1+x_1^2)$ &  \\ 
Nguyen-7 & \cite{krawiec2013approximating},\cite{worm2013prioritized,wieloch2013running,harada2014asynchronously,krawiec2014behavioral,demelo2014kaizen,lacava2015genetic,liskowski2017discovery} & $ln(x_1+1)+ln(x_1^2+1)$ & Train.: $E[0,2,20]$ / Test: $U[0,2,20]$ \\ 
Nguyen-8 & \cite{worm2013prioritized,wieloch2013running,demelo2014kaizen,liskowski2017discovery} & $\sqrt(x_1)$ &  \\ 
Nguyen-9 & \cite{worm2013prioritized,wieloch2013running,krawiec2014behavioral,demelo2014kaizen,liskowski2017discovery} & $sin(x_1)+sin(x_2^2)$ &  \\ 
Nguyen-10 & \cite{worm2013prioritized,wieloch2013running,krawiec2014behavioral,demelo2014kaizen} & $2*sin(x_1)*cos(x_2)$ &  \\ 
Abalone (ABA) & \cite{thuong2017combining} &  & 500 randomly selected instances, first column as dummy \\ 
Airfoil self-noise (AFN) & \cite{oliveira2016dispersion} & &  \\ 
Boston housing (BOH) & \cite{whigham2015examining,dick2015reexamination,lacava2016epsilon,thuong2017combining} &  &  \\ 
Combined cycle power plant (CCP) & \cite{medernach2016new} & &  \\ 
Computer hardware (CPU) & \cite{oliveira2016dispersion} &  &  \\ 
Concrete strength (CST) & \cite{medernach2016new,oliveira2016dispersion} &  &  \\ 
Energy efficiency, cooling load (ENC) & \cite{arnaldo2014multiple,arnaldo2015building,lacava2016epsilon,oliveira2016dispersion} & &  \\ 
Energy efficiency, heating load (ENH) & \cite{arnaldo2014multiple,arnaldo2015building,lacava2016epsilon,oliveira2016dispersion} &  & \\ 
Forest fires (FFR) & \cite{oliveira2016dispersion} & &  \\ 
Million song dataset (MSD) & \cite{arnaldo2015building} & & Droped \\ 
Ozone (OZO) & \cite{thuong2017combining} &  & Mean imputation
 Removed all missing value observations \\ 
Wine quality, red wine (WIR) & \cite{arnaldo2014multiple,arnaldo2015building,oliveira2016dispersion} &  &  \\ 
Wine quality, white wine (WIW) & \cite{arnaldo2014multiple,arnaldo2015building,oliveira2016dispersion} &  &  \\ 
Yacht hydrodynamics (YAC) & \cite{medernach2016new} &  &  \\ 
\end{tabular}
\end{center}
\label{tab:datasets}
\end{table*}

\subsection{Meta-attributes}

We selected a set of five meta-attributes to work with:
\begin{enumerate}
\item Number of attributes;
\item Number of instances; 
\item Target attribute skewness;
\item Standard deviation of the target attribute.
\end{enumerate}

The final meta dataset had XX attributes and XX examples.


\subsection{Data Analysis}

The meta dataset was given to a Random Forest regressor \cite{}, which generated a ranking of the most relevant meta-attributes to predict the RMSE of the problem.
We also run a dimensionality reduction method, namely PCA, and analysed the correlation of the dataset representation to its RMSE.