\documentclass[sigconf, table]{acmart}

\usepackage{array}
\usepackage{graphicx}
\usepackage{listings}
\usepackage{booktabs} % For formal tables
\usepackage{multirow} % For tables with multirow cells
\usepackage{subcaption}
\usepackage{verbatim} % Allows multi-line comments
\usepackage{mathtools}
\usepackage[linesnumbered,ruled,vlined]{algorithm2e}
\usepackage{color}

%\usepackage{flushend}

\usepackage{soul}
\usepackage{showframe}

\mathtoolsset{firstline-afterskip=0pt}

\newcommand{\etal}[0]{\emph{et al.}}

\colorlet{dark-green}{green!70!black}

\newcommand{\jf}[1]{\textcolor{orange}{[#1]}}
\newcommand{\lo}[1]{\textcolor{blue}{[#1]}} 
\newcommand{\lodel}[1]{\textcolor{blue}{[\st{#1}]}} 
\newcommand{\gi}[1]{\textcolor{red}{[#1]}}
\newcommand{\lf}[1]{\textcolor{dark-green}{[#1]}}

\newcommand{\down}{{\color[rgb]{0.7,0,0}{$\blacktriangledown$}}}
\newcommand{\up}{{\color[rgb]{0,0.45,0.1}{$\blacktriangle$}}}
\newcommand{\eq}{{\color[rgb]{0,0,0.5}{$\blacklozenge$}}}


\newcolumntype{A}{>{\raggedright\arraybackslash}m{1.65cm}}
\newcolumntype{B}{>{\raggedright\arraybackslash}m{1.75cm}}


\setcopyright{acmlicensed}

% DOI
%\acmDOI{http://dx.doi.org/10.1145/3071178.3071300}
\acmDOI{}

% ISBN
%\acmISBN{978-1-4503-4920-8/17/07}
\acmISBN{978-x-xxxx-xxxx-x/YY/MM}

% Conference
\acmConference{GECCO '18}{July 15-19, 2018}{Kyoto, Japan}
\acmYear{2018} 
\copyrightyear{2018} 


\acmPrice{15.00}

\begin{document}
	
	\title[Solving the Exponential Growth of Symbolic Regression Trees in GSGP]{Solving the Exponential Growth of Symbolic Regression Trees in Geometric Semantic Genetic Programming} 
	
	\author{Omitted due to blind review}
	\affiliation{Omitted due to blind review}
	
	\renewcommand{\shortauthors}{Ommited due to blind review}
	
	\begin{comment}
	\author{\Large Joao Francisco B. S. Martins, Luiz Otavio V. B. Oliveira, Luis F. Miranda, Felipe Casadei, Gisele L. Pappa}
	\affiliation{%
		\institution{Universidade Federal de Minas Gerais, Department of Computer Science}
		\city{Belo Horizonte} 
%		\state{Minas Gerais}
		\country{Brazil}
	}
	\email{[joaofbsm, luizvbo, luisfmiranda, casadei, glpappa]@dcc.ufmg.br}
	
	\renewcommand{\shortauthors}{Martins et al.}
	\end{comment}
	
	\begin{abstract}
		
		Advances in Geometric Semantic Genetic Programming (GSGP) have shown that this variant of Genetic Programming (GP) reaches better results than its predecessor for supervised machine learning problems, particularly in the task of symbolic regression. 
		However, by construction, geometric semantic operators generate individuals that grow exponentially with the number of generations, resulting in solutions with limited use.
		We present a new method for individual simplification named GSGP with Pruned Programs (GSGP3). 
GSGP3 works by expanding the functions generated by the geometric semantic operators. The resulting expanded function is guaranteed to be a linear combination, that, in a second step, has its repeated structures and coefficients aggregated.		
%		It expands the functions representing the offspring of geometric operators generating linear functions, which in a second step have their repeated structures and coefficients  aggregated.
		Experiments in 12 real-world datasets show that it is not only possible to create smaller and completely equivalent individuals in competitive computational time, but also to reduce the number of nodes composing them by 58 orders of magnitude, on average.
		
	\end{abstract}
	
	%
	% The code below should be generated by the tool at
	% http://dl.acm.org/ccs.cfm
	% Please copy and paste the code instead of the example below. 
	%
	\begin{CCSXML}
		<ccs2012>
		<concept>
		<concept_id>10010147.10010257.10010293.10011809.10011813</concept_id>
		<concept_desc>Computing methodologies~Genetic programming</concept_desc>
		<concept_significance>500</concept_significance>
		</concept>
		<concept>
		<concept_id>10010147.10010257.10010258.10010259.10010264</concept_id>
		<concept_desc>Computing methodologies~Supervised learning by regression</concept_desc>
		<concept_significance>300</concept_significance>
		</concept>
		</ccs2012>
	\end{CCSXML}
	
	\ccsdesc[500]{Computing methodologies~Genetic programming}
	\ccsdesc[300]{Computing methodologies~Supervised learning by regression}
	
	% We no longer use \terms command
	%\terms{Theory}
	
	\keywords{Genetic Programming; Geometric Semantic Genetic Programming; Symbolic Regression; Solution Size; Function Simplification}
	
	\maketitle
	
	\section{Introduction}

The quest for better Genetic Programming (GP) based algorithms---by overcoming known drawbacks, proposing new paradigms or hybridizing existing ones---is as important as the search for better ways of evaluating these methods and comparing them under different aspects. It is necessary to consistently and efficiently identify the scenarios where the new algorithm excels and how it compares to its predecessors.
	\section{Experimental Analysis}




	
	\begin{acks}
		Omitted due to blind review.
		
%		This work was partially supported by the following Brazilian Research Support Agencies: CNPq, FAPEMIG, CAPES and  \jf{Projeto EU}. We would like to thank Gabriel Coutinho for his valuable insights on the mathematical aspects of the problem in question.
	\end{acks}
	
	\bibliographystyle{ACM-Reference-Format}
	\bibliography{references}
	
\end{document}
